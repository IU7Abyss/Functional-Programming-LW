% !TEX root = main.tex

\section{Что будет результатом}

\biglisting{../../Problems/src/problem-1.lisp}

\begin{lstlisting}
=> Undefined function
\end{lstlisting}

\section{Функция, которая уменьшает на 10 все числа из списка-аргумента этой функции}

\biglisting{../../Problems/src/problem-2.lisp}



\section{Функция, которая возвращает  первый  аргумент списка-аргумента, который сам является непустым списком}

\biglisting{../../Problems/src/problem-3.lisp}



\section{Функция, которая выбирает из заданного списка только те числа, которые больше 1 и меньше 10}

\biglisting{../../Problems/src/problem-4.lisp}



\section{Функция, вычисляющая декартово произведение двух своих списков-аргументов}

\biglisting{../../Problems/src/problem-5.lisp}



\section{Почему так реализовано \texttt{reduce}, в чем причина?}

\begin{lstlisting}
(reduce #'+ ())  =>   0
(reduce #'+ ())  =>   0
\end{lstlisting}



\section{Функция, которая вычисляет сумму длин всех элементов}

\biglisting{../../Problems/src/problem-7.lisp}



\section{Рекурсивную версия вычисления суммы чисел заданного списка}

\biglisting{../../Problems/src/problem-8.lisp}



\section{Рекурсивная версия функции \texttt{nth}}

\biglisting{../../Problems/src/problem-9.lisp}



\section{Рекурсивную функцию  \texttt{alloddr}, которая возвращает \texttt{T}, когда все элементы списка нечётные}

\biglisting{../../Problems/src/problem-10.lisp}



\section{Рекурсивная  функция, относящаяся к хвостовой рекурсии с одним тестом завершения, которая возвращает последний элемент списка-аргумента}

\biglisting{../../Problems/src/problem-11.lisp}



\section{Рекурсивную  функция, относящуюся  к  дополняемой рекурсии с одним тестом завершения, которая вычисляет сумму всех чисел от 0 до n-аргумента функции}

\subsection{от n-аргумента функции до последнего >= 0}

\biglisting{../../Problems/src/problem-12-1.lisp}

\subsection{от n-аргумента функции до m-аргумента c шагом d}

\biglisting{../../Problems/src/problem-12-2.lisp}



\section{Рекурсивная функция, которая возвращает последнее нечётное число из числового списка}

\biglisting{../../Problems/src/problem-13.lisp}



\section{Функция которая получает как аргумент список чисел, а возвращает список квадратов этих чисел в том же порядке}

\biglisting{../../Problems/src/problem-14.lisp}



\section{функция, которая из заданного списка выбирает все нечётные числа}

\biglisting{../../Problems/src/problem-15.lisp}
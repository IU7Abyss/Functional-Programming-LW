% !TEX root = main.tex

\section{Функция, которая по своему списку-аргументу \texttt{1st} определяет является ли он палиндромом (то есть равны ли \texttt{1st} и \texttt{'(reverse 1st)})}

\biglisting{../../Problems/src/problem-1.lisp}


\section{Предикат \texttt{set-equal}, который возвращает \texttt{T}, если два его множества-аргу\-мента содержат одни и те же элементы, порядок которых не имеет значения}

\biglisting{../../Problems/src/problem-2.lisp}



\section{Функции, которые обрабатывают таблицу из точечных пар (страна . столица) и возвращают по стране --- столицу, а по столице --- страну}



\section{Функция, которая переставляет в списке-аргументе первый и последний элемент}



\section{Функция, которая переставляет в списке-аргументе два указанных своими порядковыми номерами элемента в этом списке}



\section{Функции, которые производят круговую перестановку в списке-аргументе влево и вправо}



\section{Функция, которая умножает на заданное число-аргумент все числа из заданного списка-аргумента}


\subsection{все элементы списка --- числа}


\subsection{элементы списка --- любые числа}



\section{Функция, которая их списка-аргумента, содержащего только числа, выбирает только те, которые расположены между двумя указанными гра\-ни\-ца\-ми-аргументами и возвращает их в виде списка упорядоченного по возрастанию списка чисел}
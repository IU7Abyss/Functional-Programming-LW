% !TEX root = main.tex

\section{Представить списки в виде списочных ячеек}

\paragraph{Задание 1.1} \texttt{'(open close halph)}

\begin{center}\begin{tikzpicture}
	\node [conscell] (cell1) {};
	\node [conscell, right= of cell1] (cell2) {};
	\node [conscell, right= of cell2] (cell3) {};
	
	\draw [pointer] (c2 cell1) -- (cell2);
	\draw [pointer] (c2 cell2) -- (cell3);
	\draw [pointer] (c2 cell3) -- +(1,  0) node [anchor=west] {Nil};
	\draw [pointer] (c1 cell1) -- +(0, -1) node [anchor=north] {open};
	\draw [pointer] (c1 cell2) -- +(0, -1) node [anchor=north] {close};
	\draw [pointer] (c1 cell3) -- +(0, -1) node [anchor=north] {halph};
	\foreach \i in {cell1,cell2,cell3}{\draw[fill=red] (c1 \i)circle(2pt) (c2 \i)circle(2pt);}
\end{tikzpicture}\end{center}
